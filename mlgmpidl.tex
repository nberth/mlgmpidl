\documentclass[[twoside,10pt,a4paper]{report}
\usepackage[latin1]{inputenc}
\usepackage[T1]{fontenc}
\usepackage{ae}
\usepackage{fullpage}
\usepackage{url}
\usepackage{ocamldoc}
\usepackage{makeidx}

\usepackage{fancyhdr}
\pagestyle{fancy}
\renewcommand{\headrulewidth}{0.9pt}
\renewcommand{\footrulewidth}{0pt}
\setlength{\headheight}{2.8ex}
\setlength{\footskip}{5ex}
\renewcommand{\chaptermark}[1]{ %
  \markboth{\MakeUppercase{\chaptername}\ \thechapter.\ #1}{}}
\renewcommand{\sectionmark}[1]{}
\setcounter{tocdepth}{0}
\setcounter{secnumdepth}{4}
\usepackage{color}
\definecolor{mygreen}{rgb}{0,0.6,0}

\usepackage[ps2pdf]{hyperref}

\setlength{\parindent}{0em}
\setlength{\parskip}{0.5ex}

%\usepackage{listings}
%\lstloadlanguages{Caml}

\makeindex

\title{MLGmpIDL: OCaml interface for GMP library}
\begin{document}
\maketitle


\vspace*{0.9\textheight}

All files distributed in the \textsc{MLGmpIDL} interface are
distributed under LGPL license.

Copyright (C) Bertrand Jeannet 2005-2006 for the
\textsc{MLGmpIDL} interface.

\newpage

\section*{Introduction}

This package is an \textsc{OCaml} interface for the GMP
interface, which is decomposed into 7 submodules, corresponding to C
modules:

\noindent
\begin{tabular}{l@{~:~~}l}
Mpz        & GMP integers, with side-effect semantics (as in C library) \\
Mpq        & GMP rationals, with side-effect semantics (as in C library) \\
Mpzf       & GMP integers, with functional semantics  \\
Mpqf       & GMP rationals, with functional semantics \\
Mpf        & GMP multiprecision floating-point numbers \\
Gmp\_random & GMP random number functions \\
Mpfr      & MPFR multiprecision floating-point numbers
\end{tabular}

There already exist such an interface, \textsc{mlgmp}, written by
D. Monniaux and available at
\url{http://www.di.ens.fr/~monniaux/programmes.html.en}. The
motivation for writing a new one in the APRON project were
\begin{enumerate}
\item The fact that \textsc{mlgmp} provides by default a
  functional interface to \textsc{GMP}, potentially more costly in
  term of memory allocation than an imperative interface.
  \textsc{mlgmp} provides only a relative small numbers of
  functions in an imperative version.
\item The compatibility with the \textsc{CamlIDL} tool.
  \textsc{MLGmpIDL} uses \textsc{CamlIDL}, so that other OCaml/C
  interfaces written with \textsc{CamlIDL} may reuse the
  \textsc{MLGmpIDL} \texttt{.idl} files.
\end{enumerate}

\subsection*{Requirements}

\begin{itemize}
\item GMP library (tested with version 4.0 and up)
\item MPFR library (optional, tested with version 2.2.x)
\item OCaml 3.0 or up (tested with 3.09 and 3.10)
\item Camlidl (tested with 1.05)
\end{itemize}

\subsection*{Installation}

\begin{description}
\item[Library]
Set the file Makefile.config using the Makefile.config model to your own
setting.  You might also have to modify the Makefile for executables

If you download from the subversion repository, type 'make
rebuild', which builds .ml, .mli, and \_caml.c files from .idl
files.

type 'make', possibly 'make debug', and then 'make install'

The OCaml part of the library is named gmp.cma (.cmxa, .a)
The C part of the library is named libgmp\_caml.a (libgmp\_caml\_debug.a)

'make install' installs not only .mli, .cmi, but also .idl files.

Be aware however that importing those .idl files from other .idl
files will probably request the application of SED editor with the
scripts sedscript\_caml and sedscript\_c (look at the Makefile).

\item[Interpreter and toplevel]
You may also generate runtime and toplevel with
'make gmprun', 'make gmptop'

\item[Documentation]
The documentation (currently very sketchy) is generated with ocamldoc.

'make mlapronidl.dvi'

'make html' (put the HTML files in the html subdirectoy)

\item[Miscellaneous]
'make clean' and 'make distclean' have the usual behaviour.

'make mostlyclean', in addition to 'make clean', removes the .ml,
.mli and \_caml.c files generated from .idl files.
\end{description}

\newpage

\tableofcontents

\input{ocamldoc.tex}

\appendix
\printindex
\end{document}
